\documentclass[12pt,letterpaper]{article}
\usepackage[utf8]{inputenc}
\usepackage{amsmath,amsthm,amsfonts,amssymb,amscd}
\usepackage[table]{xcolor}
\usepackage[margin=2.5cm]{geometry}
\usepackage{ragged2e}
\usepackage{graphicx}
\usepackage{multicol}
\usepackage{bussproofs}
%\usepackage[brazil]{babel}
\newlength{\tabcont}
\setlength{\parindent}{0.0in}
\setlength{\parskip}{0.05in}

\begin{document}
	
	\large \textbf{Nome}: Luís Felipe de Melo Costa Silva \\
	\textbf{Número USP}: 9297961 
    
	\begin{center}
		\LARGE \bf
		Lista de Exercícios 2 - MAC0425
	\end{center}
	
	\section*{Exercício 1}
	
	Para construirmos nossa base de conhecimento, vamos nomear os fatos:
	
	\begin{itemize}
		\item $a$: "o time joga bem"
		\item $b$: "o time ganha o campeonato"
		\item $c$: "o técnico é culpado"
		\item $d$: "os torcedores estão contentes"
	\end{itemize}
	
	Nossa base de conhecimento será, portanto:
	
	\begin{multicols}{4}
		\begin{itemize}
			\item \textbf{R1}: $a \Rightarrow b$
			\item \textbf{R2}: $\lnot a \Rightarrow c$
			\item \textbf{R3}: $b \Rightarrow d$
			\item \textbf{R4}: $\lnot d$
		\end{itemize}
	\end{multicols}
	
	\textit{1. Usando regras de inferência:}
	
	\begin{itemize}
		\item Contraposição em \textbf{R3}: \\ \textbf{R5}: $\lnot d \Rightarrow \lnot b$
		\item Modus Ponens (\textbf{R4} + \textbf{R5}): \\ \textbf{R6}: $\lnot b$
		\item Contraposição em \textbf{R1}: \\ \textbf{R7}: $\lnot b \Rightarrow \lnot a$
		\item Modus Ponens (\textbf{R6} + \textbf{R7}): \\ \textbf{R8}: $\lnot a$
		\item Modus Ponens (\textbf{R2} + \textbf{R8}): \\ \textbf{R9}: $c$		
	\end{itemize}
	
	\qed
	
	\textit{2. Usando as regras de resolução:}
	
	Para provarmos por resolução, precisamos transformar nossa base de conhecimento na Forma Normal Conjuntiva, portanto, teremos:
	
	\begin{multicols}{4}
		\begin{itemize}
			\item \textbf{R1}: $\lnot a \lor b$
			\item \textbf{R2}: $a \lor c$
			\item \textbf{R3}: $\lnot b \lor d$
			\item \textbf{R4}: $\lnot d$
		\end{itemize}
	\end{multicols}
	
	Temos que adicionar também \textbf{R5}: $\lnot c$ para fazermos a inferência baseada em resolução. Logo:
	
	\begin{prooftree}
		\AxiomC{$\lnot c$}
		\AxiomC{$a \lor c$}
		\AxiomC{$\lnot a \lor b$}
		\AxiomC{$\lnot b \lor d$}
		\AxiomC{$\lnot d$}
		\BinaryInfC{$\lnot b$}
		\BinaryInfC{$\lnot a$}
		\BinaryInfC{$c$}
		\BinaryInfC{$\bot$}
	\end{prooftree}
	
	\qed

	\section*{Exercício 8.13}
	
	\textbf{a)}
	\quad (1) $\forall s$ Cheiro$(s)$ $\Rightarrow \exists r$ Adjacente$(r, s)$ $\land$ Em$(Wumpus, r) $
	
	\quad \quad (2) $\forall s$ $\lnot$Cheiro$(s)$ $\Rightarrow \lnot \exists r$ Adjacente$(r, s)$ $\land$ Em$(Wumpus, r) $
	
	Para mostrar que as duas regras juntas equivalem a:
	\begin{center}
		$\forall s$ Cheiro$(s)$ $\Leftrightarrow \exists r$ Adjacente$(r, s)$ $\land$ Em$(Wumpus, r) $,
	\end{center}
	
	vamos chamar:
	
	\begin{center}
		\begin{itemize}
			\item de $A$: $\forall s$ Cheiro$(s)$;
			\item de $B$: $\exists r$ Adjacente$(r, s)$ $\land$ Em$(Wumpus, r) $
		\end{itemize}
	\end{center}
	
	Logo, temos que provar que $A \Leftrightarrow B$. Podemos ver que em (1) temos $A \Rightarrow B$, e que em (2) temos $\lnot A \Rightarrow \lnot B$. De (2), por Modus Ponens, podemos escrever $B \Rightarrow A$. Com isso, temos as duas expressões que nos permitem provar o que queremos. 
	
	\qed
	
	\textbf{b)} $\forall s$ Abismo$(s)$ $\Rightarrow (\forall r$ Adjacente$(r, s)$ $\Rightarrow$ Ventilada$(r))$. (1)
	
	Colocando "se não há abismo em s, então todas as localizações adjacentes a s não são ventiladas” na forma de \textbf{regra causal}, temos: $\forall s$ $\lnot$Abismo$(s)$ $\Rightarrow (\forall r$ Adjacente$(r, s)$ $\Rightarrow \lnot$ Ventilada$(r))$ (2).
	
	Chamando: 
	\begin{itemize}
		\item de A: $\forall s$ Abismo$(s)$;
		\item de B: $\forall r$ Adjacente$(r, s)$ $\Rightarrow$ Ventilada$(r)$,
	\end{itemize}
	
	temos que (1) é $A \Rightarrow B$ e (2) é $\lnot A \Rightarrow \lnot B$.
	
	Fazendo uma tabela verdade (na página 4), podemos ver que as expressões não são equivalentes.
	
	Um axioma que relaciona Adjacente($r,s$) e Abismo($r$) com o literal $\lnot$ Ventilada($s$) pode ser escrito como:
	
	\begin{center}
		$\forall r, s$ (Adjacente$(r,s) \land$ $\lnot$Abismo$(r)) \Rightarrow \lnot$Ventilada($s$)
	\end{center}
	
	\section*{Exercício 2}
	
	Para escrever um axioma de estado sucessor, temos que partir da definição:
	
	\begin{center}
		\textit{P é verdade $\Leftrightarrow$ (uma ação tornou P verdade) $\lor$ P já era
		verdade e nenhuma ação tornou P falso)} 
	\end{center}
	
	Logo, aqui teremos: 
	
	$\forall loc, s$ (Em$(Agente, loc, $resultado$(A, s)) \Rightarrow$ ViradoPara$(Agente, dir, S)) \Leftrightarrow ((\lnot$ Em$(Agente, loc, S) \land$ ViradoPara$(Agente, dir_f, S) \land A = IrParaFrente) \lor ($ Em$(Agente, loc, S) \land \lnot$ ViradoPara$(Agente, dir, S) \land A = virar)) \lor ($ Em$(Agente, loc, S) \land $ ViradoPara$(Agente, dir, S \land A \neq (VirarParaEsquerda \lor VirarParaDireita \lor IrParaFrente))$.
	
	Note que temos $A = virar$ em nosso axioma. Ele depende da direção que o agente está virado em $S$ ($dir_i$) e da direção que o agente está virado em $S'$ ($dir_f$). A tabela de correspondências está na página seguinte.
	
	
	
	
	
	\newpage 
	
	\begin{table}[]
		\centering
		\caption{Tabela do item b do exercício 8.13}
		\label{my-label}
		\begin{tabular}{|l|l|l|l|l}
			\cline{1-4}
			$A$ & $B$ & $A \Rightarrow B$ & $B \Rightarrow A$ &  \\ \cline{1-4}
			F   & F   & V                 & V                 &  \\ \cline{1-4}
			F   & V   & V                 & F                 &  \\ \cline{1-4}
			V   & F   & F                 & V                 &  \\ \cline{1-4}
			V   & V   & V                 & V                 &  \\ \cline{1-4}
		\end{tabular}
	\end{table}
	
	\begin{table}[]
		\centering
		\caption{Tabela do exercício 2}
		\label{my-label}
		\begin{tabular}{|l|l|l|ll}
			\cline{1-3}
			$dir_f$ & $dir_i$ & $virar$             &  &  \\ \cline{1-3}
			Norte   & Oeste   & VirarParaDireita    &  &  \\ \cline{1-3}
			Norte   & Leste   & VirarParaEsquerda   &  &  \\ \cline{1-3}
		    Oeste   & Sul     & VirarParaDireita    &  &  \\ \cline{1-3}
			Oeste   & Norte   & VirarParaEsquerda   &  &  \\ \cline{1-3}
			Sul     & Leste   & VirarParaDireita    &  &  \\ \cline{1-3}
			Sul     & Oeste   & VirarParaEsquerda   &  &  \\ \cline{1-3}
			Leste   & Norte   & VirarParaDireita    &  &  \\ \cline{1-3}
			Leste   & Sul     & VirarParaEsquerda   &  &  \\ \cline{1-3}
		\end{tabular}
	\end{table}
			 
\end{document}