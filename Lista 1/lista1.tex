\documentclass[12pt,letterpaper]{article}
\usepackage[utf8]{inputenc}
\usepackage{amsmath,amsthm,amsfonts,amssymb,amscd}
\usepackage[table]{xcolor}
\usepackage[margin=2.5cm]{geometry}
\usepackage{ragged2e}
\usepackage{graphicx}
\usepackage{multicol}
%\usepackage[brazil]{babel}
\newlength{\tabcont}
\setlength{\parindent}{0.0in}
\setlength{\parskip}{0.05in}

\begin{document}
	
	\large \textbf{Nome}: Luís Felipe de Melo Costa Silva \\
	\textbf{Número USP}: 9297961 
    
	\begin{center}
		\LARGE \bf
		Lista de Exercícios 1 - MAC0425
	\end{center}
	
	\section*{Exercício 3.3}
	
	\textbf{a)} Para formular um problema de busca precisamos dos seguintes itens:
	
	\begin{itemize}
		\item Estados do mundo: Cidades do mapa.
		\item Ações: Ir de uma cidade para outra.
		\item Função de transição de estado: Dado um dos amigos e a cidade onde ele está, junto com uma cidade adjacente, devolve esse amigo na cidade adjacente com custo igual a $d(i,j)$.
		\item Função de custo de caminho:
		Soma de vários termos na forma\\ $max(d(i,j), d(k,l))$, sendo $i$ e $k$ a cidade onde $A$ e $B$ estão, respectivamente, em uma iteração e $j$ e $l$, cidades adjacentes a $i$ e $k$ respectivamente, não necessariamente diferentes. Os termos tem essa forma porque um amigo espera o outro terminar o caminho antes do próximo turno começar.
		\item Um estado inicial: $A$ e $B$ nas cidades em que vivem. 
		\item Estados meta: As cidades em que $A$ e $B$ podem se encontrar.
		\item Teste de meta: Se a cidade em que $A$ está for a mesma que $B$ está, essa cidade é um estado meta.
	\end{itemize}
		
	\textbf{b)} As únicas heurísticas admissíveis são (i)$D(i,j)$ e (iii)$\frac{D(i,j)}{2}$. Uma heurística admissível é aquela que nunca ultrapassa o custo real de se alcançar a meta. Portanto, usando heurísticas menores ou iguais a uma linha reta (que é a menor distância entre dois pontos), teremos heurísticas admissíveis. 

	\section*{Exercício 3.11}
	Um \textbf{estado do mundo} é uma situação concreta no mundo real. Já uma \textbf{descrição do estado} é uma representação abstrata do mundo real que é usadas por agentes de busca. Por último, um \textbf{nó de busca} é um nó numa árvore de busca, onde a raiz é o estado inicial e os filhos de cada nó são os estados alcançáveis a partir de ações. Essa distinção é útil na modelagem dos problemas de busca. Utilizamos os estados do mundo para entender o problema e então, criamos as descrições dos estados. Com isso, procuramos a melhor abordagem para resolver o problema. Na implementação de um algoritmo para solucionar o problema, geralmente trabalhamos com grafos, por isso é útil usarmos nós de busca.
	\section*{Exercício 3.29}
	
	Sabemos que uma heurística $h$ é consistente se sua estimativa é sempre menor ou igual à distância estimada de qualquer nó vizinho até o objetivo mais o custo de chegar nesse vizinho, ou seja:
	\begin{center}
		$h(n) \leq c(n,v) + h(v)$,
	\end{center}  
	onde $v$ é um nó sucessor de $n$ e $c(x)$ é a função de custo.
	
	Uma heurística admissível $h$ é aquela que nunca ultrapassa o real custo $h^*$  de se alcançar a meta, em outros termos:
	\begin{center}
		$h(n) \leq h^*(n)$
	\end{center}
	
	Vamos provar por indução que uma heurística consistente também é admissível.
	
	\textbf{Base:} Seja $u$ um nó antecessor de $v$, que é o estado meta nesse caso. Como a heurística é consistente:
	\begin{center}
		$h(u)\leq c(u,v)+h(v)=c(u,v)+0=c(u,v)$
	\end{center}
	
	\textbf{Passo:} Seja $t$ um nó. O custo ótimo para alcançar $v$ de $t$ é $h^*(t)$.
	Ele é calculado como $min_{x\in A}(c(t, x)+h^*(x))$, onde $A$ é o conjunto de sucessores de $t$. Como a heurística é consistente, então $h(t)\leq c(t,t')+h(t')$. Além disso, como $h(t) \leq h^*(t)$ é o que estamos queremos provar, $h(t) \leq c(t, t') + h^*(t)$, e isso é verdade para todos os sucessores $t'$ do nó $t$. Em outras palavras, $h(t)\leq min_{x\in A}(c(t, x)+h^*(x))=h^*(t)$, logo, $h(t)\leq h^*(t)$. \qed
	
	\section*{Exercício Extra 1}
	\textbf{a)} Usando a BFS (busca em largura), a ordem dos nós a serem visitados será definida por uma fila (o primeiro a entrar é o primeiro a sair). Paramos de expandir os nós quando o estado meta está na borda (sétima linha). Tabela 1.
	
	\textbf{b)} Usando a DFS (busca em profundidade), a ordem dos nós a serem visitados será definida por uma pilha (o último a entrar é o primeiro a sair). Paramos de expandir os nós quando o estado meta é visitado (sétima linha). Tabela 2.
	
	\newpage
	
	\begin{table}[]
		\centering
		\caption{Expandindo com BFS}
		\label{my-label}
		\begin{tabular}{l|l}
			\textbf{Nós expandidos}                      & \textbf{Borda} \\ \cline{1-2}
			                                             & A              \\ \cline{1-2}
			A                                            & B, D, G        \\ \cline{1-2}
			A, B                                         & D, G           \\ \cline{1-2}
			A, B, D                                      & G, C, E        \\ \cline{1-2}
			A, B, D, G                                   & C, E, K, L     \\ \cline{1-2}
			A, B, C, D, G                                & E, K, L, F     \\ \cline{1-2}
			A, B, C, D, E, G                             & K, L, F, I     \\ \cline{1-2}
			A, B, C, D, E, G, K                          & L, F, I, O     \\ \cline{1-2}
			A, B, C, D, E, G, K, L                       & F, I, O, H     \\ \cline{1-2}
			A, B, C, D, E, F, G, K, L                    & I, O, H, J     \\ \cline{1-2}
			A, B, C, D, E, F, G, I, K, L                 & O, H, J, M     \\ \cline{1-2}
			A, B, C, D, E, F, G, I, K, L, O              & H, J, M        \\ \cline{1-2}
			A, B, C, D, E, F, G, H, I, K, L, O           & J, M           \\ \cline{1-2}
			A, B, C, D, E, F, G, H, I, J, K, L, O        & M              \\ \cline{1-2}
			A, B, C, D, E, F, G, H, I, J, K, L, M, O     & N              \\ \cline{1-2}
			A, B, C, D, E, F, G, H, I, J, K, L, M, N, O  &                \\
		\end{tabular}
	\end{table}

	\begin{table}[]
		\centering
		\caption{Expandindo com DFS}
		\label{my-label}
		\begin{tabular}{l|l}
			\textbf{Nós expandidos}                      & \textbf{Borda} \\ \cline{1-2}
			                                             & A              \\ \cline{1-2}
			A                                            & B, D, G        \\ \cline{1-2}
			A, B                                         & D, G           \\ \cline{1-2}
			A, B, D                                      & C, E, G        \\ \cline{1-2}
			A, B, C, D                                   & F, E, G        \\ \cline{1-2}
			A, B, C, D, F                                & I, J, E, G     \\ \cline{1-2}
			A, B, C, D, F, I                             & M, J, E, G     \\ \cline{1-2}
			A, B, C, D, F, I, M                          & N, J, E, G     \\ \cline{1-2}
			A, B, C, D, F, I, M, N                       & J, E, G        \\ \cline{1-2}
			A, B, C, D, F, I, J, M, N                    & E, G           \\ \cline{1-2}
			A, B, C, D, E, F, I, J, M, N                 & G              \\ \cline{1-2}
			A, B, C, D, E, F, G, I, J, M, N              & K, L           \\ \cline{1-2}
			A, B, C, D, E, F, G, I, J, K, M, N           & O, L           \\ \cline{1-2}
			A, B, C, D, E, F, G, I, J, K, M, N, O        & L              \\ \cline{1-2}
			A, B, C, D, E, F, G, H, I, J, K, L, M, N, O  &                \\
		\end{tabular}
	\end{table}
			 
\end{document}